\documentclass{article}

\usepackage[margin=1.5in]{geometry}

\begin{document}
My story is one of change. I barely graduated high school with a 2.1 GPA and no direction or drive. I attempted classes at the local community college for a few semesters, but finally swore off academics for what I thought was forever.

Over the course of the next five years, I worked a variety of low-skill jobs which taught me what hard work really meant. I left the country for two years to serve a volunteer mission with my religious organization. I learned to speak a foreign language while living in the Philippines and had to learn to manage my own time everyday to accomplish personal and organizational goals. While there, I became a leader of twenty other volunteers and was in charge of providing training on a bi-weekly basis. The volunteers in my purview came from a variety of backgrounds and nationalities, and it was my job to relay important news and trainings from the mission president to them. I learned much about relating to others, effective teaching, and conflict resolution. It was a great learning experience.

Upon my return to the States, I partnered with my former boss and started my own lawn care company. I drew on my past experience in similar lines of work and started making a profit after only a few months of being in business. For the next two years, I solely managed and operated my company; I learned many things about what to do (and what not to do) when you become your own boss. I came face to face with the realities of business taxes, liability claims, and customer interactions. Through it all, I maintained a large pool of returning customers and managed to find time to socialize and even meet my wife in the meantime.

My wife and I have been married for three years and have a 1.5 year-old son (as well as a girl on the way later this year). She is the reason I decided to venture into the world of academics again. She is very intelligent and values her education greatly. With her example and encouragement, I dove head first into an engineering degree at another community college. Knowing full well my own history with school, I knew I had to choose a difficult major where I would be challenged and overwhelmed. I chose engineering because it was the hardest thing I could think of. Also, engineering would allow me to do daily what the past five years had taught me I love doing: solve problems. When I see a problem, I immediately start to make causal relationships and think of solutions. For me, this is the beginning and end of what engineering entails. Engineering allows us to fill out our toolbox to be able to face problems previously unsolvable. The art of problem solving comes in when an altogether different problem is in front of us and we are required to either employ old tools in a new way or develop new tools. 

Outside of school, I have been fortunate enough to be able to perform research in the field of intelligent control under an NSF grant for undergraduates. Through this experience I learned how to conduct research in a manner which would withstand critique and repeated review. I learned how to work closely with others and build on joint skills and knowledge. I also learned how to present myself and my findings in a relevant way to a variety of audiences. This experience was invaluable to me as a growth opportunity and awakened within me a desire to explore the edge of our current capabilities and expand our horizons in the field of intelligent systems.

I carried this desire to learn and research to my co-op experience at a locally-owned research engineering firm. There I learned to apply theory to practice and put ideas to work. I helped develop a system to detect and react to thermal/visual inputs in much less than a second, managing to decrease the response time of the system by 70\%. I also developed a system to analyze high speed footage of ballistic events and extract relevant data in a user-friendly manner. Every day of my co-op experience represented more challenges and problems, many of which I had no prior experience confronting. Each time a new issue arose, I learned to ask questions, use my resources, and work with those around me to find a solution.

I have found opportunities to serve in my community as well as in my own home. At my church, I serve as a volunteer liaison between young missionaries and the church congregation. This service includes holding weekly meetings, organizing any coordination between missionaries and members, and leading goal-setting and planning operations. I find this a great opportunity to engage with a cause which means a lot to me and make a difference. I have also been a cub scout den leader for a den of Bear cubs. I was one of two leaders in charge of seven boys, planning weekly activities and helping the boys learn about becoming better citizens and members of their family. I learned more from that experience than any of those boys for sure! I learned that when one disciplines a child, one needs to do so with love and understanding or else the relationship will become damaged and stop progressing. This is definitely valuable knowledge as I begin raising my own child!
\end{document}